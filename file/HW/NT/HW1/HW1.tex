%hw-4.tex
%%%%%%%%%%%%%%%%%%%%%%%%%%%%%%%%%%%%%%%%%%%%%%%%%%%%%%%%%%%%%%%%%%%%%%
%%%%%%            Math 403 Homework \#4, Spring 2025            %%%%%%
%%%%%%                                                          %%%%%%
%%%%%%                 Instructor: Ezra Miller                  %%%%%%
%%%%%%%%%%%%%%%%%%%%%%%%%%%%%%%%%%%%%%%%%%%%%%%%%%%%%%%%%%%%%%%%%%%%%%
\documentclass[11pt]{article}

\oddsidemargin=17pt \evensidemargin=17pt
\headheight=9pt     \topmargin=-36pt
\textheight=564pt   \textwidth=433.8pt
\parskip=1ex        %voffset=-6ex

\usepackage{amsmath, amsthm, amssymb,graphicx,color,enumitem}%,setspace%,mathrsfs%,hyperref
\setlist[enumerate]{parsep=1.5ex plus 4pt,topsep=0ex plus 4pt,itemsep=0ex}
  \setlist[itemize]{parsep=1.5ex plus 4pt,topsep=-1ex plus 4pt,itemsep=-1.33ex}

\leftmargini=3.5ex
\leftmarginii=3.5ex

%%%%
\usepackage{thmtools}
\declaretheoremstyle[
spaceabove=6pt, spacebelow=6pt,
headfont=\normalfont\bfseries,
notefont=\mdseries, notebraces={(}{)},
bodyfont=\normalfont,
postheadspace=1em,
numberwithin=section
]{exstyle}
\declaretheoremstyle[
spaceabove=6pt, spacebelow=6pt,
headfont=\normalfont\bfseries,
notefont=\mdseries, notebraces={(}{)},
bodyfont=\normalfont,
postheadspace=1em,
headpunct={},
qed=$\blacktriangleleft$,
numbered=no
]{solstyle}
\declaretheorem[style=exstyle]{example}
\declaretheorem[style=exstyle]{nonexample}
\declaretheorem[style=solstyle]{solution}

%for \marginpar to fit optimally
%hoffset=-1.02in
\setlength\marginparwidth{2.2in}
\setlength\marginparsep{1mm}
\newcommand\red[1]{\marginpar{\linespread{.85}\sf%
		   \vspace{-1.4ex}\footnotesize{\color{red}#1}}}
\newcommand\score[1]{\marginpar{\vspace{-2ex}\color{blue}{#1/3}}\hspace{-1ex}}
\newcommand\extra[1]{\marginpar{\color{blue}{#1/1}}\hspace{-1ex}}
\newcommand\total[2]{\marginpar{\colorbox{yellow}{\huge #1/#2}}}
\newcommand\collab[1]{\marginpar{\vspace{-11ex}\colorbox{yellow}{#1/3}}\hspace{-1ex}}
\newcommand\magenta[1]{\colorbox{magenta}{\!\!#1\!\!}}
\newcommand\yellow[1]{\colorbox{yellow}{\!\!#1\!\!}}
\newcommand\green[1]{\colorbox{green}{\!\!#1\!\!}}
\newcommand\cyan[1]{\colorbox{cyan}{\!\!#1\!\!}}
\newcommand\rmagenta[2][0ex]{\red{\vspace{#1}\magenta{\phantom{:}}\,: #2}}
\newcommand\ryellow[2][0ex]{\red{\vspace{#1}\yellow{\phantom{:}}\,: #2}}
\newcommand\rgreen[2][0ex]{\red{\vspace{#1}\green{\phantom{:}}\,: #2}}
\newcommand\rcyan[2][0ex]{\red{\vspace{#1}\cyan{\phantom{:}}\,: #2}}

%For separated lists with consecutive numbering
\newcounter{separated}

\newcommand{\excise}[1]{}
\newcommand{\comment}[1]{{$\star$\sf\textbf{#1}$\star$}}

%%%%%%%%%%%%%%%%%%%%%%%%%%%%%%%%%%%%%%%%%%%%%%%%%%%
%                                                 %
% TO ENABLE GRADING, DO NOT ALTER ABOVE THIS LINE %
%                                                 %
%%%%%%%%%%%%%%%%%%%%%%%%%%%%%%%%%%%%%%%%%%%%%%%%%%%

%new math symbols taking no arguments
\newcommand\0{\mathbf{0}}
\newcommand\CC{\mathbb{C}}
\newcommand\FF{\mathbb{F}}
\newcommand\NN{\mathbb{N}}
\newcommand\QQ{\mathbb{Q}}
\newcommand\RR{\mathbb{R}}
\newcommand\ZZ{\mathbb{Z}}
\newcommand\bb{\mathbf{b}}
\newcommand\kk{\Bbbk}
\newcommand\mm{\mathfrak{m}}
\newcommand\xx{\mathbf{x}}
\newcommand\yy{\mathbf{y}}
\newcommand\GL{\mathit{GL}}
\newcommand\FL{{\mathcal F}{\ell}_n}
\newcommand\SO{\mathit{SO}}
\newcommand\into{\hookrightarrow}
\newcommand\minus{\smallsetminus}
\newcommand\goesto{\rightsquigarrow}

%redefined math symbols taking no arguments
\newcommand\<{\langle}
\renewcommand\>{\rangle}
\renewcommand\iff{\Leftrightarrow}
\renewcommand\implies{\Rightarrow}

%to explain about LaTeX commands:
\newcommand\command[1]{\texttt{$\backslash$#1}}

%new math symbols taking arguments
\newcommand\ol[1]{{\overline{#1}}}

%redefined math symbols taking arguments
\renewcommand\mod[1]{\ (\mathrm{mod}\ #1)}

%roman font math operators
\DeclareMathOperator\im{im}

%for easy 2 x 2 matrices
\newcommand\twobytwo[1]{\left[\begin{array}{@{}cc@{}}#1\end{array}\right]}

%for easy column vectors of size 2
\newcommand\tworow[1]{\left[\begin{array}{@{}c@{}}#1\end{array}\right]}


%%%%%%%%%%%%%%%%%%%%%%%%%%%%%%%%%%%%%%%%%%%%%%%%%%%%%%%%%%%%%%%%%%%%%%
\begin{document}%%%%%%%%%%%%%%%%%%%%%%%%%%%%%%%%%%%%%%%%%%%%%%%%%%%%%%
%%%%%%%%%%%%%%%%%%%%%%%%%%%%%%%%%%%%%%%%%%%%%%%%%%%%%%%%%%%%%%%%%%%%%%


\title{\textbf{Number Theory Homework \#1} \textit{Spring 2025}\\\normalsize
	Instructor: \textit{Chan Ieong Kuan}\\[-2ex]}
\author{Solutions by: \textit{\textbf{Haoyu Zhu}}\\[2ex]}
\date{}
\maketitle

\vspace{-5ex}%

\noindent
\textsc{Exercises}% covers Lectures 14 - 17

\begin{enumerate}
\section{UFD}
\item\ 
Exercise 1.1:
\begin{proof}
$(a,b)|(b,r)$: $r = a-qb$, so $(a,b)|r$. Additionally, $(a,b)|b$.\par
$(b,r)|(a,b)$: by virtue of $a = qb+r$, then proceeds similar asforehead.
\end{proof}
\item\ 
Exercise 1.3:
\begin{itemize}
    \item (187, 221) = (187, 34) = (34, 17) = 17.
    \item (6188, 4709) = (4709, 1479) = (1479, 272) = (272, 119) = (119, 34) = (34, 17) = 17.
    \item (314,159) = (159,155) = (155, 4) = 1
\end{itemize}
\item\
Exercise 1.16:
\begin{proof}
    Since $\ZZ$ is a unique factorization domain and $(u,v) = 1$, one can decompose $u,v$ with $0 \leq n_1 \leq n_2, e_i > 0$, and $p_i$ primes: 
\begin{eqnarray*}
    u = \prod_{i=1}^{n_1}p_i^{e_i};
    v = \prod_{i=n_1+1}^{n_2}p_i^{e_i};
    uv = \prod_{i=1}^{n_2}p_i^{e_i}
\end{eqnarray*}
In light of $uv = a^2$, $e_i$ must all be even. Consequently, $u,v$ are squares.

\end{proof}
\item\ Exercise 1.21
\begin{proof}
    Without loss of generality, assume $ord_p\,a \leq ord_p\,b$, and write $r = ord_p\,(a+b), s = ord_p\,a, t = ord_p\,b$ satisfying
    \[a = p^sq_a,\, b = p^tq_b, (a+b)= p^s(q_a + p^{(t-s)}q_b),\] where
    $q_a, q_b$ are relatively prime to $p$. Because $p^s$ divides $a+b$, 
    the inequality holds.\par
    Moreover, when $t > s$, \[q_a + p^{t-s}q_b \equiv q_a \mod{p}\], so $s$ is the maximal power of $p$ factorizing $a+b$,
    thereby the equlity.
\end{proof}
\item\ Exercise  1.23
\begin{proof}
    Suppose that $a,b$ are both odd, then $c^2$ is even.
    As a result, $c$ is even and $4|c^2 = a^2 + b^2$, but $a^2, b^2$ are equivalent to 1 modulo 4. This is impossible. Consequently, 
    one of $a,b$ should be even, say $a$. For $a,b,c$ are pairly coprime, $b,c$ are odd numbers. Now $a^2 = (c-b)(c+b)$, and 4 divides both sides. One obtains: 
    \[\left(\frac{a}{2}\right)^2 = \frac{(c-b)}{2}\frac{(c+b)}{2}\] Again, by virtue of $(b,c)=1$,
    there exist $x,y \in \ZZ$ such that $xb+yc=1$, then \[(x+y)(\frac{(c+b)}{2})+ (y-x)\frac{(c-b)}{2} = 1.\] Hence $(\frac{(c-b)}{2}, \frac{(c+b)}{2})=1$.
    According to Exercise 1.16, both $\frac{(c+b)}{2}$ and $\frac{(c-b)}{2}$ are squares which are coprime. Thereby the existence of $u,v$ as stated in the problem.\par
    Conversely, when $a = 2uv, b= v^2-u^2, c = v^2+u^2$, then 、\[a^2 + b^2 = 4u^2v^2 + u^4 - 2u^2v^2 +v^4 = c^2\].

\end{proof}
\newpage
\section{Arithmetic Functions}
\subsection*{1. Number of Divisors Function: \(\nu(n)\)}
The function \(\nu(n)\) counts the number of positive divisors of \(n\):
\[
\nu(n) = \sum_{d \mid n} 1
\]
\subsection*{2. Sum of Divisors Function: \(\sigma(n)\)}
The function \(\sigma(n)\) calculates the sum of all positive divisors of \(n\):
\[
\sigma(n) = \sum_{d \mid n} d
\]
\subsection*{3. Generalized Sum of Divisors Function: \(\sigma_s(n)\)}
For a real or complex number \(s\), the function \(\sigma_s(n)\) is defined as:
\[
\sigma_s(n) = \sum_{d \mid n} d^s
\]
\subsection*{4. Euler's Totient Function: \(\phi(n)\)}
Euler's totient function \(\phi(n)\) counts the number of integers up to \(n\) that are coprime to \(n\):
\[
\phi(n) = \left| \{k \in \mathbb{N} \mid 1 \leq k \leq n, \gcd(k, n) = 1\} \right|
\]

\subsection*{5. Mobius Function: \(\mu(n)\)}
The Möbius function \(\mu(n)\) is defined as:
\[
\mu(n) = 
\begin{cases}
1 & \text{if } n = 1, \\
(-1)^k & \text{if } n \text{ is the product of } k \text{ distinct primes}, \\
0 & \text{if } n \text{ has a squared prime factor}.
\end{cases}
\]

\subsection*{6. Riemann Zeta Function: \(\zeta(s)\)}
The Riemann zeta function \(\zeta(s)\) is defined for complex numbers \(s\) with \(\Re(s) > 1\) as:
\[
\zeta(s) = \sum_{n=1}^\infty \frac{1}{n^s}
\]
\item\ Exercise  2.10
\begin{proof}
    For every coprime pair $m,n$, once $d\mid mn$, $d$ can be uniquely factored into $d=d_1d_2$, where $d_1\mid m, d_2\mid n$.
    Provided $(m,n) = 1$,  one has 
\[
g(mn) = \sum_{d \mid mn} f(d) = \sum_{\substack{d_1 \mid m \\ d_2 \mid n}} f(d_1 d_2) = \sum_{\substack{d_1 \mid m \\ d_2 \mid n}} f(d_1) f(d_2) = \left( \sum_{d_1 \mid m} f(d_1) \right) \left( \sum_{d_2 \mid n} f(d_2) \right) = g(m) g(n)
\]
\end{proof}
\item\ Exercise  2.12
\begin{proof}
    Since $\phi(n), \mu(n)$ are multiplicative, the combinations of their products or quotients remain multiplicative. In terms of Exercise 2.10, 
    the three Arithmetic Functions are multiplicative, too. As a result, they are determined completely by its value on prime powers. For any 
    $n = \prod_{i=1}^{k}{p_i}^{e_i}$, applying multiplicative $f$ one obtains $f(n) = \prod_{i=1}^{k}f({p_i}^{e_i})$. Now we only need to set $n = p^e$.
    \begin{itemize}
        \item $\sum_{d|p^e}{\mu(d)\phi(d)} = \mu(1)\phi(1) + \mu(p)\phi(p) = 1 - (p-1) = 2 - p.$
        \item $\sum_{d|p^e}{\mu(d)^2\phi(d)^2} = \mu(1)^2\phi(1)^2 + \mu(p)^2\phi(p)^2 = 1 + (p-1)^2 = p^2 - 2p +2.$
        \item $\sum_{d|p^e}{\mu(d)/\phi(d)} = \mu(1)/\phi(1) + \mu(p)/\phi(p) = 1 - (p-1)^{-1}.$
    \end{itemize}
    The formulas for arbitrary positive integrers follow trivially.
\end{proof}
\item\ Exercise  2.22
\begin{proof}
    For convenience, we define  
    \[
    f(n) = \sum_{(t,n)=1} t.
    \]  
    We start with the sum of all elements in \(\{1,2,\dots,n\}\):  
    \[
    \sum_{i=1}^{n} i = \frac{(1+n)n}{2}.
    \]  
    Next, we partition the elements based on their greatest common divisor with \( n \), rewriting the sum as:  
    \[
    \sum_{i=1}^{n} i = \sum_{d \mid n} \sum_{\substack{(t,n)=d}} t.
    \]  
    Using the substitution \( t = d t' \) where \( (t', n/d) = 1 \), we obtain:  
    \[
    \sum_{\substack{(t,n)=d}} t = d \sum_{\substack{(t',n/d)=1}} t' = d f(n/d).
    \]  
    Thus, we conclude:  
    \begin{align*}
        &\frac{(1+n)n}{2} = \sum_{d \mid n} d f(n/d),\\
        \iff\quad &\frac{(1+n)n}{2} = \sum_{d \mid n} (n/d) f(d),\\
        \iff\quad &(1+n) = \sum_{d \mid n} \frac{2 f(d)}{d}.
    \end{align*}
    
    Define \( g(n) = \frac{2 f(n)}{n} \). We want to show that $g(n) = \phi(n)$. 
    Writing \( n \) in its prime factorization as \( n = \prod_{i=1}^{k} p_i^{e_i} \), we obtain:  
    \begin{align*}
        &(1+n) = g \ast I(n),\\
        \iff\quad &g(n) = (\mu \ast I) \ast g(n) = \mu \ast (I \ast g)(n) = \sum_{d \mid n} \mu(d) \left( 1 + \frac{n}{d} \right),\\
        \iff\quad &g(n) = (1+n) - \sum_{i=1}^{k} \left( 1 + \frac{n}{p_i} \right) + \dots + (-1)^k \left( 1 + \frac{n}{\prod_{i=1}^{k} p_i} \right).
    \end{align*}
    
    Note that $1 + \frac{n}{p}$ is the number of elements in $\{0,1,2,\ldots,n\}$ divided by $p$ for $p\mid n$. By the principle of inclusion-exclusion, $g(n)$ equals to the number of elements in $\{0,1,2,\ldots,n\}$ that are coprime to $n$.
    We conclude that \( g(n) = \phi(n) \).
    
\end{proof}
\item\ Exercise  2.25
\begin{proof}
    We start with the Euler product formula for the Riemann zeta function:

    \[
        \prod_{p} \left( 1 - \frac{1}{p^s} \right)^{-1} = \prod_{p} \sum_{i=1}^{\infty} \left( \frac{1}{p^s} \right)^i.
    \]
    
    Taking the product over all primes and expanding the infinite series, we obtain:
    
    \begin{eqnarray}\label{zeta}
        \prod_{p} \left( 1 - \frac{1}{p^s} \right)^{-1} = \prod_{p} \sum_{i=1}^{\infty} \frac{1}{p^{is}} = \sum\prod_{i=1}^{N_n} \frac{1}{(p_i^s)^{e_i}}.
    \end{eqnarray}
    
    Next, we recover the Dirichlet series definition of the Riemann zeta function:
    
    \begin{eqnarray}
        \zeta(s) = \sum_{n=1}^{\infty} \frac{1}{n^s}.
    \end{eqnarray}
    
    Now, consider the alternative product expansion since each \( n \) can be uniquely factored as \( n = p_1^{e_1} p_2^{e_2} \cdots p_k^{e_k} \):
    \begin{eqnarray}
        \sum_{n=1}^{\infty} \frac{1}{n^s} = \sum_{n=1}^{\infty} \frac{1}{\prod_{i=1}^{N_n} (p_i^s)^{e_i}}.
    \end{eqnarray}
    
    
    We see that this expansion matches the right-hand side of the Euler product in equation \eqref{zeta}, establishing the connection between the two representations of \( \zeta(s) \).
    
\end{proof}
\item\ Exercise  2.26
\begin{proof}
\begin{itemize}
    \item{Verification of \( \zeta(s)^{-1} = \sum \frac{\mu(n)}{n^s} \)}

Starting from the Euler product representation of \( \zeta(s) \):

\[
\zeta(s) = \prod_{p} \left( 1 - \frac{1}{p^s} \right)^{-1}.
\]

Taking the reciprocal:

\[
\zeta(s)^{-1} = \prod_{p} \left( 1 - \frac{1}{p^s} \right).
\]

Expanding each term as an infinite product:

\[
\prod_{p} \left( 1 - \frac{1}{p^s} \right) = \sum \frac{(-1)^k}{(\prod_{i=1}^{k}{p_i^s})}= \sum_{n=1}^{\infty} \frac{\mu(n)}{n^s}.
\]

\item{Verification of \( \zeta(s)^2 = \sum \frac{\nu(n)}{n^s} \)}

Squaring the Euler product results in a double sum:
\[
\zeta(s)^2 = \sum_{m=1}^{\infty} \sum_{n=1}^{\infty} \frac{1}{(mn)^s}.
\]
We introduce the function \( \nu(n) \), which counts the number of ways to write \( n \) as a product of two factors \( m \) and \( n \):
\[
\nu(n) = \sum_{d|n} 1.
\]
Thus, we can partition the sum above in terms of the value of $mn$, write $M$, and obtain:
\[
    \sum_{m=1}^{\infty} \sum_{n=1}^{\infty} \frac{1}{(mn)^s} = 
    \sum_{M=1}^{\infty} \sum_{n\mid M} \frac{1}{(M)^s} = 
    \sum_{M=1}^{\infty} \frac{\nu(M)}{M^s}.
\]
\item{Verification of \( \zeta(s) \zeta(s-1) = \sum \frac{\sigma(n)}{n^s} \)}

Starting from the product representation:

\[
\zeta(s) \zeta(s-1) = \left( \sum_{m=1}^{\infty} \frac{1}{m^s} \right) \left( \sum_{n=1}^{\infty} \frac{1}{n^{s-1}} \right).
\]

Expanding the double sum:

\[
\sum_{m=1}^{\infty} \sum_{n=1}^{\infty} \frac{1}{m^s n^{s-1}} 
\overset{(M = mn)}{=}
\sum_{M=1}^{\infty} \sum_{n \mid M} \frac{n}{M^s} 
= \sum_{M=1}^{\infty} \frac{\sum_{n \mid M} n}{M^s} 
= \sum_{M=1}^{\infty} \frac{\sigma(M)}{M^s}.
\]

where \( \sigma(M) = \sum_{d|M} d \) is the sum of divisors function.
\end{itemize}
\end{proof}
\end{enumerate}


%%%%%%%%%%%%%%%%%%%%%%%%%%%%%%%%%%%%%%%%%%%%%%%%%%%%%%%%%%%%%%%%%%%%%%
\end{document}%%%%%%%%%%%%%%%%%%%%%%%%%%%%%%%%%%%%%%%%%%%%%%%%%%%%%%%%
%%%%%%%%%%%%%%%%%%%%%%%%%%%%%%%%%%%%%%%%%%%%%%%%%%%%%%%%%%%%%%%%%%%%%%


%%% Various Emacs customizations:
%%% Local Variables:
%%% fill-column: 70
%%% indent-tabs-mode: t
%%% TeX-electric-sub-and-superscript: nil
%%% TeX-brace-indent-level: 0
%%% LaTeX-indent-level: 0
%%% LaTeX-item-indent: 0
%%% TeX-master: t
%%% End: