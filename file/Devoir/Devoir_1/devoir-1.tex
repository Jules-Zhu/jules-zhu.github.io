\documentclass[a4paper,12pt]{article}
\usepackage[french]{babel}
\usepackage[T1]{fontenc}
\usepackage[utf8]{inputenc}
\usepackage{geometry}
\geometry{left=2cm,right=2cm,top=0cm,bottom=2cm}

\title{Devoir 1 : Dialogues}
\author{Jules et Emma}
%\date{}

\begin{document}

\maketitle
\subsection*{Scène 1}
\textbf{Jules :} Lopez ; \textbf{Emma :} Agence Artès.

\bigskip

\textbf{Vincent Lopez :} Bonjour.

\textbf{Agence Artès :} Bonjour, monsieur ! Emma à l'appareil. C'est de la part de qui ?

\textbf{Vincent Lopez :} Je m'appelle Vincent Lopez. Je vous appelle au sujet de votre annonce pour de la figuration au festival de théâtre de rue d'Aurillac.

\textbf{Agence Artès :} Bonjour Vincent ! Merci de votre intérêt. Pouvez-vous vous présenter ?

\textbf{Vincent Lopez :} Bien sûr ! J'ai 17 ans, je mesure 1m80, j'ai les yeux marron et les cheveux blonds. Je suis disponible en juin et juillet pour participer au festival.

\textbf{Agence Artès :} Très bien ! Vous êtes disponible pour un rendez-vous mercredi prochain à 14h30 ?

\textbf{Vincent Lopez :} D'accord, je note. Merci beaucoup !

\textbf{Agence Artès :} Est-ce que vous avez un numéro de téléphone ?

\textbf{Vincent Lopez :} Oui, bien sûr ! C'est le 04 74 05 30 96.

\textbf{Agence Artès :} Parfait, Vincent. Nous vous attendons donc mercredi prochain à 14h30. Bonne journée et à bientôt !

\textbf{Vincent Lopez :} Merci, à mercredi ! Au revoir !

\bigskip
\subsection*{Scène 2}

\textbf{Emma :} Bonjour.

\textbf{Jules :} Bonjour ! Département de mathématiques, Jules à l'appareil. Que puis-je faire pour vous ?

\textbf{Emma :} Je m'appelle Emma. Je vous appelle au sujet de l'annonce pour le poste d'assistante d'enseignement.

\textbf{Jules :} Très bien, Emma ! Pouvez-vous me parler un peu de votre profil ?

\textbf{Emma :} Bien sûr ! Je suis douée pour menacer mes camarades de classe de faire leurs devoirs. De plus, je suis douée pour distinguer l'écriture et je peux voir s'il y a des camarades de classe qui aident d'autres personnes à signer des formulaires d'inscription.

\textbf{Jules :} C'est très bien ! Nous organisons des entretiens pour les candidats. Seriez-vous disponible pour un rendez-vous la semaine prochaine ?

\textbf{Emma :} Oui, bien sûr ! Quel jour et à quelle heure ?

\textbf{Jules :} Nous pouvons vous proposer jeudi à 10h30. Cela vous convient-il ?

\textbf{Emma :} Oui, parfait ! Pouvez-vous me confirmer l'adresse exacte ?

\textbf{Jules :} Bien sûr ! L'entretien aura lieu au bureau 302, bâtiment des sciences. Vous devrez apporter votre CV et vos relevés de notes.

\textbf{Emma :} D'accord, je note. Merci beaucoup !

\textbf{Jules :} De rien ! Avant de terminer, pourriez-vous me donner votre numéro de téléphone ?

\textbf{Emma :} Oui, c'est le 86 73 10 67 01.

\textbf{Jules :} Parfait, Emma. Donc c'est entendu pour jeudi prochain à 10h30. Au revoir !

\textbf{Emma :} Merci, à jeudi ! Au revoir !

\end{document}
